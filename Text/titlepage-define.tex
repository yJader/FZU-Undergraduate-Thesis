% -------------------------------------------------------
% -                       自定义命令                      -%
% -------------------------------------------------------
\makeatletter

\newcommand{\contenttitle}[1]{\def\@contenttitle{#1}}
\newcommand{\contenttitleSecond}[1]{\def\@contenttitleSecond{#1}}
\newcommand{\institute}[1]{\def\@institute{#1}}
\newcommand{\major}[1]{\def\@major{#1}}
\newcommand{\studentid}[1]{\def\@studentid{#1}}


\newcommand{\showtitle}{\@title}
\newcommand{\showcontenttitle}{\@contenttitle}
\newcommand{\showcontenttitleSecond}{\@contenttitleSecond}
\newcommand{\showinstitute}{\@institute}
\newcommand{\showmajor}{\@major}
\newcommand{\showstudentid}{\@studentid}




% 设置默认值,防止未定义时报错
\providecommand{\@contenttitle}{记得填写标题!}
\providecommand{\@contenttitleSecond}{\@empty}
\providecommand{\@institute}{记得填写学院!}
\providecommand{\@major}{记得填写专业!}
\providecommand{\@studentid}{记得填写学号!}



\newcolumntype{M}[1]{>{\centering\arraybackslash}m{#1}} % 水平居中的m列
\newcolumntype{B}[1]{>{\centering\arraybackslash}b{#1}} % 水平居中的b列

\makeatother

% -------------------------------------------------------
% -                       自定义封面命令                  -%
% -------------------------------------------------------
\makeatletter
\renewcommand{\maketitle}{
    \pagestyle{empty}\fancyhf{}
    \begin{titlepage}
        % 复制word的图片, 尽管测量了距离, 还是有一点误差
        % \begin{textblock*}{\textwidth}(\dimexpr (\paperwidth - \textwidth) / 2 \relax, 3.85cm) %距离页面顶部距离3.85cm
        %     \begin{figure}[H]
        %         \centering
        %         \includegraphics[height=2.25cm, keepaspectratio]{TemplateAssets/fzu1.png}
        %         \hspace{0.55cm}
        %         \includegraphics[height=2.25cm, keepaspectratio]{TemplateAssets/fzu2.png}
        %         \hspace{0.55cm}
        %         \includegraphics[height=2.25cm, keepaspectratio]{TemplateAssets/fzu3.png}
        %         \hspace{0.74cm}
        %         \includegraphics[height=2.25cm, keepaspectratio]{TemplateAssets/fzu4.png}
        %     \end{figure}
        % \end{textblock*}

        % 矢量图版本, 使用校徽的矢量图, 对每个字添加了等高且等距的轮廓后, 放大到word封面相同高度
        % 但是发现学校下发的word封面可能并没有严格等距...
        \begin{textblock*}{\textwidth}(\dimexpr (\paperwidth - \textwidth) / 2 \relax, 4.10cm) %距离页面顶部距离4.10cm
            \begin{figure}[H]
                \centering
                \includegraphics[height=1.85cm, keepaspectratio]{TemplateAssets/fzu1.pdf}
                \hspace{0.56cm}
                \includegraphics[height=1.85cm, keepaspectratio]{TemplateAssets/fzu2.pdf}
                \hspace{0.56cm}
                \includegraphics[height=1.85cm, keepaspectratio]{TemplateAssets/fzu3.pdf}
                \hspace{0.56cm}
                \includegraphics[height=1.85cm, keepaspectratio]{TemplateAssets/fzu4.pdf}
            \end{figure}
        \end{textblock*}

        % \newcolumntype{M}[1]{>{\centering\arraybackslash}m{#1}}
        % \newcolumntype{B}[1]{>{\centering\arraybackslash}b{#1}}

        \begin{textblock*}{\textwidth}(\dimexpr (\paperwidth - \textwidth) / 2 \relax, 8.08cm)
            \begin{center}
                %中文题目:宋体1号字
                % \linespread{1.5}\selectfont %设置行距
                \tbf \yihao \hspace{1.2cm}\@title
            \end{center}
        \end{textblock*}

        \begin{textblock*}{\textwidth}(\dimexpr (\paperwidth - \textwidth) / 2 \relax, 10.6cm)
            {
                    %学科专业等:宋体3号字
                    \song \sanhao
                    \setlength{\tabcolsep}{0.12cm}
                    \begin{center}
                        \begin{tabular}[b]{B{3cm} B{9.18cm}}
                            {题\hspace{1.1cm}目:} & {\@contenttitle}       \\\Xcline{2-2}{1pt}       \\[0.24cm]
                            {}                  & {\@contenttitleSecond} \\  \Xcline{2-2}{1pt}                   \\[0.24cm]
                            {姓\hspace{1.1cm}名:} & {\@author}             \\\Xcline{2-2}{1pt}    \\[0.45cm]
                            {学\hspace{1.1cm}号:} & {\@studentid}          \\\Xcline{2-2}{1pt}    \\[0.45cm]
                            {学\hspace{1.1cm}院:} & {\@institute}          \\\Xcline{2-2}{1pt}    \\[0.45cm]
                            {专\hspace{1.1cm}业:} & {\@major}              \\\Xcline{2-2}{1pt}    \\[0.45cm]
                            {年\hspace{1.1cm}级:} & {2021级}                \\\Xcline{2-2}{1pt}    \\[0.45cm]
                        \end{tabular}

                        \vspace{-0.17cm}

                        \begin{tabular}[b]{B{4.1cm} B{6.03cm} B{2cm}}
                            {校内指导教师:} &
                            % 在这里插入校内导师签名图片
                            % 替换为你的签名图片, e.g., Fig/signature_internal.png
                            \raisebox{-0.3\height}{\includegraphics[height=0.8cm, keepaspectratio]{TemplateAssets/example-signature.pdf}}
                                      & {(签名)} \\\Xcline{2-2}{1pt}\\[0.45cm]  % 空白行
                            {校外指导教师:} &
                            % 在这里插入校外导师签名图片, 如没有, 可以注释该行
                            % 替换为你的签名图片, e.g., Fig/signature_external.png
                            % \raisebox{-0.3\height}{\includegraphics[height=0.8cm, keepaspectratio]{example-image}} 
                                      & {(签名)} \\\Xcline{2-2}{1pt} \\[0.45cm]  % 空白行
                        \end{tabular}
                    \end{center}

                    \begin{center}
                        \song \sanhao
                        \hspace{8em}\@date
                    \end{center}
                }
        \end{textblock*}
        \ %
    \end{titlepage}
    \newpage
}
\makeatother