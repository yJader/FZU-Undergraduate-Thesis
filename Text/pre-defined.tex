%-------------------------------------------------------%
%-                          包                         -%
%-------------------------------------------------------%
\usepackage{CJK,CJKnumb}
\usepackage{url}
\usepackage[backend=biber,style=numeric-comp,sorting=none,gbpub=false,bibstyle=gb7714-2015,citestyle=gb7714-2015,
  gbpunctin=false,gbnamefmt=lowercase,
]{biblatex}

\usepackage{amsmath, amsfonts, amssymb}
\usepackage{booktabs} %三线表
\usepackage{threeparttable}
\usepackage{multicol}
\usepackage{multirow}
\usepackage{fancyhdr}
\usepackage{latexsym}
\usepackage{mathrsfs}
\usepackage{wasysym}
\usepackage{enumerate}
\usepackage{titlesec}
\usepackage{epstopdf}
\usepackage{caption}
\usepackage{graphicx, subfig} %图包
\usepackage{float}
\usepackage{setspace}
\usepackage{array}
\usepackage{fancyhdr} %页眉页脚包
\usepackage{fontspec} %英文字体包
\usepackage{titletoc} %设置目录格式
\usepackage{geometry} %设置页边距
% \usepackage{circledsteps}%enumerate带圈的序号
\usepackage{pifont}
\usepackage{adjustbox}

\usepackage[absolute,overlay]{textpos}
% \usepackage{tocloft}
\usepackage{hyperref} % href蓝链, 可以删除
\usepackage{pdfpages} % 插入pdf文件
\usepackage{makecell} % 

% 论文尺寸规格为A4(210×297mm)。每一面的上方(天头)和左侧(订口)应分别留边25mm,下方(地脚)和右侧(切口)应分别留边20mm。
\geometry{
  a4paper,
  left=20mm,
  right=20mm,
  top=25mm,
  bottom=20mm,
  bindingoffset=5mm,
  % showframe,         % 保持开启以观察页面效果
}
%论文尺寸为A4,全文页面设置为:上下2.54cm,左右3.17cm,含有大量图表可微调。页边距0.5cm
% \geometry{a4paper,left=3.17cm,right=3.17cm,top=2.54cm,bottom=2.54cm,bindingoffset=0.5cm} 

\setCJKmainfont{simsun.ttc} %文中中文为宋体
\setmainfont{Times New Roman} %文中英文为新罗马
\setlength{\baselineskip}{20pt} %行距为固定值20磅


%---------------------------------------------------------------------------%
%-                        字体设置                               -%
%---------------------------------------------------------------------------%
\setCJKfamilyfont{song}{simsun.ttc}
\setCJKfamilyfont{fs}{simfang.ttf}
\setCJKfamilyfont{kai}{simkai.ttf}
\setCJKfamilyfont{hei}{simhei.ttf}

\newcommand{\song}{\CJKfamily{song}}    % 宋体   (simsun.ttc)
\newcommand{\fs}{\CJKfamily{fs}}        %仿宋体  (simfs.ttf)
\newcommand{\kai}{\CJKfamily{kai}}      % 楷体   (simkai.ttf)
\newcommand{\hei}{\CJKfamily{hei}}      % 黑体   (simhei.ttf)

%设置加粗体
%加粗楷体
\setCJKfamilyfont{kaitib}{simkai.ttf}[AutoFakeBold]
\newcommand{\kaitib}{\CJKfamily{kaitib}}
%加粗仿宋体
\setCJKfamilyfont{fangsongti}{simfang.ttf}[AutoFakeBold]
\newcommand{\fangsongti}{\CJKfamily{fangsongti}}
%加粗黑体
\setCJKfamilyfont{heitib}{simhei.ttf}[AutoFakeBold]
\newcommand{\heitib}{\CJKfamily{heitib}}
%加粗宋体
\setCJKfamilyfont{songtib}{simsun.ttc}[AutoFakeBold]
\newcommand{\songtib}{\CJKfamily{songtib}}
%正文加粗快捷键
\newcommand{\tbf}{\bfseries \songtib}
%---------------------------------------------------------------------------%
%-                          字号设置                            -%
%---------------------------------------------------------------------------%
\newcommand{\chuhao}{\fontsize{42pt}{\baselineskip}\selectfont}
\newcommand{\xiaochuhao}{\fontsize{36pt}{\baselineskip}\selectfont}
\newcommand{\yihao}{\fontsize{26pt}{\baselineskip}\selectfont}
\newcommand{\xiaoyihao}{\fontsize{24pt}{\baselineskip}\selectfont}
\newcommand{\erhao}{\fontsize{22pt}{\baselineskip}\selectfont}
\newcommand{\xiaoerhao}{\fontsize{18pt}{\baselineskip}\selectfont}
\newcommand{\sanhao}{\fontsize{16pt}{\baselineskip}\selectfont}
\newcommand{\xiaosanhao}{\fontsize{15pt}{\baselineskip}\selectfont}
\newcommand{\sihao}{\fontsize{14pt}{\baselineskip}\selectfont}
\newcommand{\xiaosihao}{\fontsize{12pt}{\baselineskip}\selectfont}
\newcommand{\wuhao}{\fontsize{10.5pt}{\baselineskip}\selectfont}
\newcommand{\xiaowuhao}{\fontsize{9pt}{\baselineskip}\selectfont}
\newcommand{\liuhao}{\fontsize{7.5pt}{\baselineskip}\selectfont}
\newcommand{\qihao}{\fontsize{5.5pt}{\baselineskip}\selectfont}
%---------------------------------------------------------------------------%
%-                       自定义枚举列表格式                                -%
%---------------------------------------------------------------------------%
% 重定义第一级 enumerate 标签格式为 (1)
\renewcommand{\labelenumi}{(\theenumi)}
% 重定义第二级 enumerate 标签格式为 ①、②、③……
\renewcommand{\theenumii}{\arabic{enumii}} % 使用阿拉伯数字计数
\renewcommand{\labelenumii}{\textcircled{\theenumii}} % 将数字放入圈中
%---------------------------------------------------------------------------%
%-                               Chinesization                             -%
%---------------------------------------------------------------------------%
\newtheorem{theorem}{\hskip 2em定理}[section]
\newtheorem{definition}{\hskip 2em定义}[section]
\newtheorem{exam}{\hskip 2em例}[section]
\newtheorem{note}{\hskip 2em注}[section]
\newtheorem{proof}{{\it \hskip 2em\textbf{证明}}}[section]
\newtheorem{solution}{{\it \hskip 2em\textbf{解}}}
\renewcommand{\tablename}{\song 表}
\renewcommand{\figurename}{\song 图}
\renewcommand{\refname}{\centerline {参考文献}}
\renewcommand{\contentsname}{\fontsize{15.75pt}{\baselineskip}\selectfont \textbf{目~~~~录}}
\renewcommand{\thefootnote}{\arabic{footnote}}
\renewcommand{\theequation}{\thesection-\arabic{equation}}
\renewcommand{\thefigure}{\thesection-\arabic{figure}}
\renewcommand{\thetable}{\thesection-\arabic{table}}

%---------------------------------------------------------------------------%
%-                     公式随章编号                               -%
%---------------------------------------------------------------------------%
\date{}
\makeatletter
\renewcommand*\l@chapter[2]{%
  \ifnum \c@tocdepth >\m@ne
    \addpenalty{-\@highpenalty}%
    \vskip 1.0em \@plus\p@
    \setlength\@tempdima{1.5em}%
    \begingroup
    \parindent \z@ \rightskip \@pnumwidth
    \parfillskip -\@pnumwidth
    \leavevmode \bfseries
    \advance\leftskip\@tempdima
    \hskip -\leftskip
    #1\nobreak\leaders\hbox{$\m@th
        \mkern \@dotsep mu\hbox{-}\mkern \@dotsep
        mu$}\hfill \nobreak\hb@xt@\@pnumwidth{\hss #2}\par
    \penalty\@highpenalty
    \endgroup
  \fi}

% \date{}
% \numberwithin{equation}{section}
% \numberwithin{figure}{section}
% \numberwithin{table}{section}

%---------------------------------------------------------------------------%
%-                              Tabular Option                             -%
%---------------------------------------------------------------------------%
\newcommand{\tabincell}[2]{\begin{tabular}{@{}#1@{}}#2\end{tabular}}
%---------------------------------------------------------------------------%
%-                              Caption Option                             -%
%---------------------------------------------------------------------------%
\captionsetup[figure]{labelsep=space}
\captionsetup[table]{labelsep=space}
%------------------------------------------------------%
%-                    文献来源                          -%
%-------------------------------------------------------%
\addbibresource{Text/reference.bib}
%------------------------------------------------------%
%-                Page Control                        -%
%------------------------------------------------------%
% \topmargin=0cm \oddsidemargin=0.5cm \textwidth=15cm
% \textheight=22cm

%------------------------------------------------------%
%-                 目录格式配置                         -%
%------------------------------------------------------%
%目录正文:宋体5号,一级节标题加粗,行距17磅,段前0行,段后0行
%一级标题(如: 第一章 引言),首行不缩进
%二级标题(如:1.1研究背景及意义),首行缩进2个汉字
%三级标题(如:1.1.1研究背景),首行缩进4个汉字
\titlecontents{section}[0em]
{\fontsize{12pt}{\baselineskip}\selectfont}
{\hspace*{3em}\contentslabel{3em}\ }%
{}
{\titlerule*[0.7pc]{$\cdot$}\contentspage\hspace*{1em}}%

\titlecontents{subsection}[0em]
{\fontsize{12pt}{\baselineskip}\selectfont}
{\hspace*{3em}\contentslabel{1.5em}\ }% \hspace*{3em}效果不好
{}
{\titlerule*[0.7pc]{$\cdot$}\contentspage\hspace*{1em}}%

\titlecontents{subsubsection}[0em]
{\fontsize{12pt}{\baselineskip}\selectfont}
{\hspace*{4em}\contentslabel{2em}\ }%
{}
{\titlerule*[0.7pc]{$\cdot$}\contentspage\hspace*{1em}}%
\renewcommand{\baselinestretch}{1.1}


% 确保公式编号按章节划分,并包含章节号
% 例如,第一章的第一个公式是 (1-1) 或 (1.1) 取决于 \theequation 的定义
\numberwithin{equation}{section}

% 步骤 1: 将 \theequation 定义为纯数字编号部分 (例如: 1-1)
\renewcommand{\theequation}{\arabic{section}-\arabic{equation}}

% 步骤 2: 修改 \tagform@ 以自定义公式标签的显示格式
\makeatletter % 允许访问宏包内部命令
% #1 代表由 \theequation 生成的纯数字编号 (例如 "1-1")
% 我们将其格式化为 "公式 (#1)"
\def\tagform@#1{\maketag@@@{\text{公式 }(#1)\@@italiccorr}}
\makeatother