%页眉页脚设置
\pagestyle{fancy}
\fancyhead{} %clear all fields
\fancyhead[CE]{\songti \wuhao \showtitle} %偶数页
\fancyhead[CO]{\songti \wuhao \showcontenttitle \showcontenttitleSecond} %奇数页
\fancyfoot[C]{\xiaowuhao \thepage} %设置页脚
\renewcommand{\headrulewidth}{1pt} %页眉与正文之间的水平线粗细
\renewcommand{\footrulewidth}{0pt}

\section{绪论}
\setcounter{section}{1}
\setcounter{subsection}{0}
\setcounter{table}{0}
\setcounter{figure}{0}
\setcounter{equation}{0}
\setcounter{definition}{0}

\subsection{基本使用说明}
文档编译方式:

xelatex-biber-xelatex*2(离线版本)

xelatex (overleaf)

每章节前面的图、表、定义、公式的计数器要归0。

图、表、公式使用与引用方式按\LaTeX 规定。

\begin{equation}
    a=b+c \label{eq-1}
\end{equation}

\begin{definition}
    这是一个定义
\end{definition}

参考文献引用\cite{kai1979prospect}。

交叉引用: 公式(\ref{eq-1})。

\begin{table}[h!]
    \centering
    \caption{常用 \LaTeX{} 表格语法示例\\(顺便演示一下表格标题和标签的使用)}
    \label{tab:latex_example}
    \renewcommand{\arraystretch}{1.3}
    \begin{tabular}{cll}
        \toprule
        \textbf{功能} & \textbf{语法}                   & \textbf{说明} \\
        \midrule
        合并单元格       & \verb|\multicolumn{2}{c}{内容}| & 横向合并2列      \\
        多行单元格       & \verb|\multirow{2}{*}{内容}|    & 纵向合并2行      \\
        斜体          & \verb|\textit{斜体}|            & 斜体文本        \\
        粗体          & \verb|\textbf{粗体}|            & 粗体文本        \\
        插入水平线       & \verb|\hline|                 & 插入横线        \\
        表格标题        & \verb|\caption{标题}|           & 设置表格标题      \\
        表格标签        & \verb|\label{tab:xxx}|        & 交叉引用用标签     \\
        设置列宽        & \verb|p{3cm}|                 & 固定列宽3cm     \\
        \bottomrule
    \end{tabular}
\end{table}

\begin{figure}[H] % H表示强制放在当前位置, 你也可以使用htbp等选项
    \centering
    \includegraphics[width=0.3\textwidth]{TemplateAssets/FZU-motto.pdf}
    \caption{这是一个图片示例}
    \label{fig:latex_example}
\end{figure}

\subsection{各节一级标题}

这是内容

\subsubsection{各节二级标题}
你是内容


\subsection{字体样式}

{\song 宋体 \kai 楷体 \hei 黑体 \fs 仿宋}

{\tbf 加粗宋体 \kaitib 加粗楷体 \heitib 加粗黑体 \fangsongti 加粗仿宋体}



\newpage